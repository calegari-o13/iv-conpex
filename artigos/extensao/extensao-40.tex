\documentclass[article,12pt,onesidea,4paper,english,brazil]{abntex2}

\usepackage{lmodern, indentfirst, color, graphicx, microtype, lipsum}			
\usepackage[T1]{fontenc}		
\usepackage[utf8]{inputenc}		

\setlrmarginsandblock{2cm}{2cm}{*}
\setulmarginsandblock{2cm}{2cm}{*}
\checkandfixthelayout

\setlength{\parindent}{1.3cm}
\setlength{\parskip}{0.2cm}

\SingleSpacing

\begin{document}
	
	\selectlanguage{brazil}
	
	\frenchspacing 
	
	\begin{center}
		\LARGE CRIMES CIBERNÉTICOS: HACKERS, INFECTANTES SOCIAIS\footnote{Trabalho realizado dentro da área de Conhecimento CNPq/CAPES: Ciências Humanas no IFRO/Campus Cacoal.}
		
		\normalsize
		Vinícius de Oliveira Engelhardt\footnote{Bolsista (Ensino Médio), vinicius.engelhardt@gmail.com, FRO/Campus Cacoal.} 
		Rennan Iuri Braz Ramos\footnote{Colaborador (Ensino Médio), rennan.brazramos@gmail.com, IFRO/Campus Cacoal.} 
		Sirley Leite Freitas\footnote{Orientadora, sirley.freitas@ifro.edu.br, IFRO/Campus Cacoal.} 
	\end{center}
	
	\noindent O uso da internet já não é mais uma novidade em nosso cotidiano, simultaneamente com o benefício desse serviço surgiram os crimes virtuais. Os primeiros crimes cibernéticos surgiram nos Estados Unidos (EUA). Os crimes virtuais têm como definição usual, qualquer delito em que tenha sido utilizado um computador, uma rede ou um dispositivo de hardware, onde são feitos ataques como a disseminação de vírus, distribuição de material pornográfico, fraudes bancárias, violação de propriedade intelectual ou mera invasão de sites para deixar mensagens difamatórias como forma de insulto a outras pessoas. O Projeto, Crimes Cibernéticos: Hackers, infectantes sociais, com cunho de pesquisa, o que e quais são os crimes cibernéticos e outras questões relacionadas ao tema, tem por objetivo de gerar uma reflexão pessoal do tema em tela e informar os alunos do IFRO/Campus Cacoal e a comunidade local os perigos da internet, pois, qualquer usuária da rede mundial de computadores, mal informado, pode praticar algum tipo de crime virtual. Dentre aspectos reconhecidos no projeto, foram feitas pesquisa para conhecimento do assunto abordado e a construção científica necessitante empregada. Com auxílio do orientador foi realizada uma pesquisa com intuito de fazer um referencial teórico do tema. Foram descobertas as principais formas de crimes cibernéticos, as formas mais comuns, o público alvo dos golpistas internautas e finalidade do crime, para então entender o motivo ao qual o individuo realizou o delito. Também são muito comuns crimes de pedofilia, discriminação, racismo, injúria racial e de ameaças, esses cometidos pela população dita normal. O projeto ainda se encontra em fase de desenvolvimento agora a próxima etapa será promover debates dentro do espaço escolar e realizar um seminário aberto a comunidade escolar para fomentar a discussão e relevância do tema visando informar sobre os perigos de rede mundial de computares quando mal utilizada e quais os tipos de crime mais comuns realizados via internet.
	
	\vspace{\onelineskip}
	
	\noindent
	\textbf{Palavras-chave}: Internet. Disseminação. Crime.
	
\end{document}
