\documentclass[article,12pt,onesidea,4paper,english,brazil]{abntex2}

\usepackage{lmodern, indentfirst, color, graphicx, microtype, lipsum}			
\usepackage[T1]{fontenc}		
\usepackage[utf8]{inputenc}		

\setlrmarginsandblock{2cm}{2cm}{*}
\setulmarginsandblock{2cm}{2cm}{*}
\checkandfixthelayout

\setlength{\parindent}{1.3cm}
\setlength{\parskip}{0.2cm}

\SingleSpacing

\begin{document}
	
	\selectlanguage{brazil}
	
	\frenchspacing 
	
	\begin{center}
		\LARGE CICLO DE PALESTRAS (HISTORIA DE RONDÔNIA): UMA ABORDAGEM DAS TRANSFORMAÇÕES SÓCIO-ESPACIAIS\footnote{Trabalho realizado dentro da (área de Conhecimento CNPq/CAPES: Ciências Humanas – História)}
		
		\normalsize
		Waldelaine Rodrigues Hoffmann\footnote{(modalidade), waldelaine-hoffmann@homail.com,  Campus Ji-Paraná} 
		Maria Isabel Lima Barriviera \footnote{(a), mariaisabellimabarriviera@gmail.com , Campus Ji-Paraná} 
		Lourival Inácio Filho\footnote{Orientador(a):  lourival.filho@ifro.edu.br,  Campus Ji-Paraná}  
	\end{center}
	
	\noindent Este projeto faz parte das atividades desenvolvidas pelo Núcleo Informatizado de Memória e Pesquisa do IFRO (NIMPI), em consonância com as políticas públicas atuais, o projeto Ciclo de Palestras buscou proporcionar aos alunos do Instituto Federal de Educação, Ciência e Tecnologia, Campus Ji-Paraná, uma melhor compreensão das transformações dos espaços locais enquanto processo histórico que produziram sentidos, cultura e territorialidade. Foram realizadas três palestras de história de Rondônia, sendo estas “Forte Príncipe da beira a “modernidade” na selva” (05/11/2015) e “A Estrada de Ferro Madeira–Mamoré: imagens, vídeos e história (19/11/2015), ambas no auditório do IFRO; “Do Território Federal a criação do estado de Rondônia (11/11/2015) no Teatro Dominguinhos. Foram palestrantes o Professor Lourival Inácio Filho e o jornalista e escritor Anísio Gorayed (convidado). O público alvo foram alunos e alunas dos segundos anos dos cursos Técnicos Integrados ao Ensino Médio de Florestas e Química e público externo. Puderam-se desenvolver atividades de ensino, pesquisa e extensão, quebrando a rotina das salas de aulas. As turmas envolvidas no ciclo de palestras e debates informais, não só participaram ativamente com pesquisas prévias e perguntas embasadas, como também no envolvimento da organização do evento (recebendo o público externo e realizando inscrições). Durante as palestras se viu ainda apresentações culturais de dança e música desenvolvidas pelos discentes, ampliando a visão para além do monopólio do texto escrito, sem negar a importância deste, porém, ao despertar sentidos e emoções, fez-se possível envolver os alunos em atividades de longa duração, sem sermos monótonos. Destacando-se a grande participação do público que lotou o auditório, bem como o Teatro Dominguinhos.
	
	\vspace{\onelineskip}
	
	\noindent
	\textbf{Palavras-chave}: Palestras. Rondônia. Dinâmicas.
	
\end{document}
