\documentclass[article,12pt,onesidea,4paper,english,brazil]{abntex2}

\usepackage{lmodern, indentfirst, color, graphicx, microtype, lipsum}			
\usepackage[T1]{fontenc}		
\usepackage[utf8]{inputenc}		

\setlrmarginsandblock{2cm}{2cm}{*}
\setulmarginsandblock{2cm}{2cm}{*}
\checkandfixthelayout

\setlength{\parindent}{1.3cm}
\setlength{\parskip}{0.2cm}

\SingleSpacing

\begin{document}
	
	\selectlanguage{brazil}
	
	\frenchspacing 
	
	\begin{center}
		\LARGE ESTATUTO DA CRIANÇA E DO ADOLESCENTES (ECA) E SUA IMPORTÂNCIA NO MEIO SOCIAL \footnote{Trabalho realizado dentro da área de Conhecimento CNPq/CAPES: Ciências Humanas com financiamento do DEPEX/IFRO/Campus Cacoal.}
		
		\normalsize
		Vitor Fernandes\footnote{Bolsista (Ensino Médio), vitorfernandes001@gmail.com, IFRO/Campus Cacoal.} 
		Weslei Douglas de Melo\footnote{Colaborador (Ensino Médio), wesleidouglas@gmail.com, IFRO/Campus Cacoal.} 
		Sirley Leite Freitas\footnote{Orientadora, sirley.freitas@ifro.edu.br, IFRO/Campus Cacoal.}  
	\end{center}
	
	\noindent Após vários debates e mobilização, tanto da sociedade quanto dos governantes, houve um consenso de que tanto a infância como a adolescência deveria ser protegida pela sociedade das várias formas de violência, e que todos são responsáveis por garantir o desenvolvimento íntegro das crianças e dos adolescentes. A partir de então foram criadas diversas maneiras de proteger a integridade das crianças e dos adolescentes do Brasil. O primeiro é a Constituição Federal Brasileira de 1988, que determina as prioridades absolutas, que são a proteção da infância e a garantia de seus direitos, não só por partes do Estado, mais também de seus pais ou responsáveis. Com base e fundamentada no artigo 227 da Constituição Federal de 1988, foi criada a Lei nº 8.069 do dia 13 de julho de 1990, o Estatuto da Criança e do Adolescentes (ECA). O ECA estabelece uma série de normas e regras, os deveres dos pais e responsáveis em relação à criança e ao adolescente, bem como os direitos da criança e do adolescente do Brasil. No ECA estão estabelecidos os direitos fundamentais das crianças e dos adolescentes e as sanções para quando houver o descumprimento desses preceitos legais. Há também a definição do que vem a ser ato infracional e as punições para quando esses são cometido. Assim, o presente projeto de extensão visou difundir o debate sobre o tema em tela e conscientizar os alunos do IFRO/Campus Cacoal e também a comunidade local da importância do respeito às leis e neste caso em específico o respeito as crianças e adolescentes. Para tanto foram promovidos debates dentro do espaço escolar sobre o assunto e realizado palestras dentro do IFRO/Campus Cacoal e em escola da rede Estadual de ensino no município de Cacoal. Foi possível compreender durante o desenvolvimento do projeto que as leis criadas para as crianças e adolescentes não vem só para dar vantagens ou punições para eles, mais sim mostrar que criança e adolescentes devem ser cuidados e protegido para poderem desenvolver-se com segurança podendo se torna uma pessoa ética e um cidadão feliz cumpridor de seus direitos e deveres.
	
	\vspace{\onelineskip}
	
	\noindent
	\textbf{Palavras-chave}: Criança e adolescente. Direitos. Cidadania.
	
\end{document}
