\documentclass[article,12pt,onesidea,4paper,english,brazil]{abntex2}

\usepackage{lmodern, indentfirst, color, graphicx, microtype, lipsum}			
\usepackage[T1]{fontenc}		
\usepackage[utf8]{inputenc}		

\setlrmarginsandblock{2cm}{2cm}{*}
\setulmarginsandblock{2cm}{2cm}{*}
\checkandfixthelayout

\setlength{\parindent}{1.3cm}
\setlength{\parskip}{0.2cm}

\SingleSpacing

\begin{document}
	
	\selectlanguage{brazil}
	
	\frenchspacing 
	
	\begin{center}
		\LARGE OFICINAS PARA O ENSINO DE LÓGICA E MATEMÁTICA\footnote{Trabalho realizado dentro da área de Conhecimento CNPq/CAPES: Ciências Exatas e da Terra com financiamento do Instituto Federal de Rondônia – IFRO.}
		
		\normalsize
		Roberto Simplício Guimarães\footnote{Orientador, roberto.simplicio@ifro.edu.br, Campus Vilhena.} 
		Clayton Ferraz Andrade\footnote{Co-orientadora, claudia.prates@ifro.edu.br} 
		Claudia Aparecida Prates\footnote{Co-orientador, clayton.andrade@ifro.edu.br} 
		Claydaiane Ferraz Andrade\footnote{Co-orientadora, claydaiane@hotmail.com}
		Rodrigo Stiz\footnote{Co-orientador, rodrigo.stiz@ifro.edu.br, Campus Vilhena.} 
	\end{center}
	
	\noindent Este trabalho tem como finalidade estimular o pensamento lógico e matemático dos alunos no processo de aprendizagem, exigindo poder de abstração, planejamento e disciplina mental, onde, por meio da observação dos problemas do mundo real  os modela em uma linguagem de programação. O projeto “Oficinas para o ensino de Lógica e Matemática”, foi realizado parcialmente, sendo  dividido em duas etapas de execução, a primeira foi referente à produção  das caixas de lógica e a segunda à realização de oficinas.   A caixa de lógica é um instrumento que fornece ao aluno de maneira lúdica a aprendizagem de Lógica e Matemática. Trata-se de um conjunto de 12 pequenas lâmpadas coloridas (led’s) que podem ser, ou não, acionados por quatro botões distintos e que um pequeno autofalante (buzzer) pode, ou não, reproduzir uma série de sons monofônicos, mas de frequências variadas. À medida que o aluno realiza os exercícios propostos em sala de aula, ele se sente desafiado a realizar novos experimentos e de forma lúdica, desenvolver sua abstração e ordenar seus pensamentos de forma lógica. É essencial que os alunos compreendam e raciocinem de forma clara o que está sendo proposto e não apenas decorem a aplicação de fórmulas matemáticas. É necessário que as habilidades de desenvolvimento de estratégias para solução de problemas complexos sejam aprimoradas. O projeto possibilitou a confecção de caixas lógicas, as quais serão usadas no desenvolvimento de oficinas com os alunos dos 1ºs anos do curso Técnico em Informática, preferencialmente os que apresentaram dificuldades de aprendizagem nas disciplinas Lógica de Programação e Matemática, durante o primeiro semestre de 2016. Pesquisando-se as possíveis causas dos baixos índices de aproveitamento, observou-se que uma grande parcela dos alunos encontra dificuldades na passagem do raciocínio intuitivo, ainda que matemático, para uma linguagem de programação. O projeto também considerou as pesquisas bibliográficas, discussões em âmbito da instituição e conversas com profissionais da área de informática, eletrônica, matemática e pedagogia.  As oficinas serão desenvolvidas, uma vez na semana, em contraturno.  A utilização da caixa lógica  nas oficinas possibilitará  aos alunos dos 1ºs anos do curso Técnico em Informática a melhoria do rendimento nas disciplinas  de Lógica de Programação e Matemática.
	
	\vspace{\onelineskip}
	
	\noindent
	\textbf{Palavras-chave}: Aprendizagem. Lógica Matemática. Curso Técnico.
	
\end{document}
