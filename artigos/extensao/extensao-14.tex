\documentclass[article,12pt,onesidea,4paper,english,brazil]{abntex2}

\usepackage{lmodern, indentfirst, color, graphicx, microtype, lipsum}			
\usepackage[T1]{fontenc}		
\usepackage[utf8]{inputenc}		

\setlrmarginsandblock{2cm}{2cm}{*}
\setulmarginsandblock{2cm}{2cm}{*}
\checkandfixthelayout

\setlength{\parindent}{1.3cm}
\setlength{\parskip}{0.2cm}

\SingleSpacing

\begin{document}
	
	\selectlanguage{brazil}
	
	\frenchspacing 
	
	\begin{center}
		\LARGE CAPACITAÇÃO EM PROCESSAMENTO DE ALIMENTOS OBJETIVANDO A SUSTENTABILIDADE E A GERAÇÃO DE RENDA PARA O POVO INDÍGENA PAITER SURUÍ\footnote{Trabalho realizado dentro da (área de Conhecimento CNPQ/CAPES: Ciência e Tecnologia de Alimentos) com financiamento do IFRO/DEPEX/CAMPUS CACOAL.}
		
		\normalsize
		Amanda Fernandes Gonçalves\footnote{Bolsista Extensão, amanda\_pbro@hotmail.com, Campus Cacoal.} 
		Sidnéia de Lima Nunes\footnote{Bolsista Extensão, sidneya.cacoal02@gmail.com, Campus Cacoal.} 
		Iramaia Grespan Ferreira\footnote{Orientadora, iramaia.grespan@ifro.edu.br, Campus Cacoal} 
		 
	\end{center}
	
	\noindent Segundo a ONU (2010) os povos indígenas sofrem com a vulnerabilidade, pois existe a desigualdade: na educação, na distribuição de terras, no emprego, nos níveis de segurança alimentar, de mortalidade infantil e violência, além do impacto de novos problemas como a questão ambiental e a migração, deste modo, este projeto visa preencher parte dessa lacuna atendendo demandas da educação e reduzindo a desigualdade. Os Suruís de Rondônia se autodenominam Paiter, que significa "gente de verdade, nós mesmos", hoje totalizam segundo dados do levantamento socioeconômico 2010, uma população aproximada de 1200 pessoas. Vivem majoritariamente em 24 aldeias, distribuídos em 215 famílias ao longo e nas proximidades das fronteiras da Terra Indígena Sete de Setembro. O país ainda está longe de ter um ensino adequado para os seus povos das florestas, são inúmeras as dificuldades encontradas e Rondônia não fica longe desse panorama. A falta de escolas nas aldeias, a pouca organização da comunidade escolar para compreender a problemática educacional indígena, assim como, a carência de professores qualificados, são algumas dessas dificuldades. Hoje a Associação Metareílá, representada pelo seu líder Almir Suruí, busca uma alternativa para a preservação das florestas que sobraram, através de projetos sustentáveis, mantendo as riquezas da floresta, além do reflorestamento de suas terras. Esse projeto de extensão capacita 30 mulheres indígenas nos aspectos referentes ao processamento de alimentos, através de um Curso de Formação Inicial e Continuada – FIC. Inicialmente visitou-se as aldeias para conhecer a produção agrícola, bem como, compreender a realidade vivenciada nas aldeias, para assim tornar as aulas mais atrativas e interessantes. O curso FIC é ministrado na Associação Metareilá do Povo Suruí e objetiva a sustentabilidade, a segurança alimentar e uma alternativa de geração de renda para os Paiter Suruí, pois as mulheres estão identificando novas possibilidades econômicas na sua produção agrícola, onde se destaca a produção de milho, café, mandioca, cará, batata, inhames, banana, mamão, amendoim, castanha. Este projeto é uma forma de interação valiosa, pois estimula a valorização dos produtos do povo Paiter Suruí.
	
	\vspace{\onelineskip}
	
	\noindent
	\textbf{Palavras-chave}: Capacitação. Sustentabilidade. Suruí.
	
\end{document}
