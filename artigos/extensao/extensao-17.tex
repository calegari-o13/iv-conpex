\documentclass[article,12pt,onesidea,4paper,english,brazil]{abntex2}

\usepackage{lmodern, indentfirst, color, graphicx, microtype, lipsum}			
\usepackage[T1]{fontenc}		
\usepackage[utf8]{inputenc}		

\setlrmarginsandblock{2cm}{2cm}{*}
\setulmarginsandblock{2cm}{2cm}{*}
\checkandfixthelayout

\setlength{\parindent}{1.3cm}
\setlength{\parskip}{0.2cm}

\SingleSpacing

\begin{document}
	
	\selectlanguage{brazil}
	
	\frenchspacing 
	
	\begin{center}
		\LARGE UTILIZAÇÃO DE PAPEL RECICLADO PARA A GERAÇÃO DE RENDA EXTRA\footnote{Trabalho realizado dentro da (área de Conhecimento CNPq/CAPES: Química) com financiamento da PROEX/IFRO.}
		
		\normalsize
		João Lucas S. Viana\footnote{Bolsista (Ensino Médio), joaolucas1007@gmail.com, Campus Ji-Paraná.} 
		Juliana Alves Rodrigues\footnote{Colaborador(a), juliana2013ifro@gmail.com, Campus Ji-Paraná.} 
		Kelly Cristina Souza Borges\footnote{Bolsista (Ensino Superior), kellycris.quimi@gmail.com, Campus Ji-Paraná.} 
		Fabyana Aparecida Soares\footnote{Orientadora, fabyana.soares@ifro.edu.br, Campus Ji-Paraná} 
	\end{center}
	
	\noindent Uma das grandes preocupações de ambientalistas, governos e população em geral é o impacto ambiental causado pelo descarte incorreto de resíduos gerados nos mais diversos setores da sociedade. Um destes resíduos é o papel que é descartado em escritórios, fábricas, lojas, etc. O papel descartado terá como destinos finais lixões e aterros sanitários e leva cerca de 6 meses para degradar-se. O destino e tratamento final do lixo não é uma obrigação apenas do poder público e sim de toda a sociedade.  A reciclagem é uma forma de reaproveitamento da matéria-prima e é de extrema importância para o meio ambiente. Como sabemos, o papel é produzido através da celulose de determinados tipos de árvores. Quando reciclamos o papel, estamos contribuindo com o meio ambiente. Não podemos esquecer também, que a reciclagem de papel gera renda para pessoas no Brasil que atuam, principalmente, em cooperativas de catadores e recicladores de papel. A reciclagem será o resultado de uma série de atividades pelas quais materiais que seriam descartados, são coletados, separados e processados para serem usados como matéria-prima na manufatura de novos produtos. A comercialização dos produtos obtidos a partir da reciclagem de papel é um incentivo à inclusão social, geração de trabalho e renda. Foi idealizado um projeto de extensão, com a finalidade de ministrar oficinas de reciclagem de papel para pais e alunos do Campus Ji-Paraná com condição socioeconômica vulnerável e o público atendido na Associação Resgate de Vidas Ernesta A G Bernardi. Primeiramente essas oficinas foram realizadas com os alunos do IFRO Campus Ji-Paraná, que mostraram interesse em produzir os papeis reciclados. O objetivo do projeto é de desenvolver no cidadão uma conscientização ambiental sobre o desperdício de papel, bem como a produção de produtos que poderão gerar rendas para complementação da renda familiar. Para produzir o papel reciclado são utilizados materiais de uso comum, tendo como matéria prima os papeis descartados, cola ou fibras. O papel é triturado e posto de molho na água, após algum tempo é batido e misturado com a cola ou fibras. Durante a oficina os alunos aprenderam as técnicas para produzir o papel reciclado. Em seguida utilizaram o papel produzido como matéria prima para criação de caixas de presente e cartões, que futuramente podem ser comercializados e com isso gerar uma renda extra para a família.
	
	\vspace{\onelineskip}
	
	\noindent
	\textbf{Palavras-chave}: . Educação Ambiental. Papel Reciclado.
	
\end{document}
