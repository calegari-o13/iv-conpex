\documentclass[article,12pt,onesidea,4paper,english,brazil]{abntex2}

\usepackage{lmodern, indentfirst, color, graphicx, microtype, lipsum}			
\usepackage[T1]{fontenc}		
\usepackage[utf8]{inputenc}		

\setlrmarginsandblock{2cm}{2cm}{*}
\setulmarginsandblock{2cm}{2cm}{*}
\checkandfixthelayout

\setlength{\parindent}{1.3cm}
\setlength{\parskip}{0.2cm}

\SingleSpacing

\begin{document}
	
	\selectlanguage{brazil}
	
	\frenchspacing 
	
	\begin{center}
		\LARGE O PODER DA PALAVRA\footnote{Trabalho realizado dentro da (área de Conhecimento CNPq/CAPES: Letras) com financiamento do DEPEX – Departamento de Extensão – Campus Colorado do Oeste.}
		
		\normalsize
		Guibson Dimas da Silva Oliveira\footnote{Colaborador (aluno monitor), email, Campus Colorado do Oeste} 
		Gisely Storch do Nascimento Santos\footnote{Orientador, gisely.storch@ifro.edu.br, Campus Colorado do Oeste} 
		Moisés José Rosa Souza\footnote{Co-orientador Moisés José Rosa Souza, moises.souza@ifro.edu.br, Campus Colorado do Oeste.} 
	
	\end{center}
	
	\noindent Escrever é uma arte, mas, nestes tempos modernos, é, antes de tudo, uma necessidade. "A arte de escrever precisa assentar numa atividade preliminar já radicada, que parte do ensino escolar e de um hábito de leitura inteligente conduzido”, conforme assevera Câmara (1997). O projeto de Extensão intitulado “O Poder da Palavra”, desenvolvido pelos docentes da área de linguagens do Campus Colorado do Oeste visa propiciar momentos de estudo e trocas de experiências aos professores de Língua Portuguesa da rede Estadual, Municipal e Privada do município de Colorado do Oeste, abrangendo todos os níveis de ensino. Além de promover um concurso de produção de textos entre alunos das escolas participantes, com vistas a motivar e fomentar práticas discursivas em sala de aula. O projeto está em sua terceira edição. Para a edição de 2016 o tema trabalhado foi “Como alimentar sem destruir”. Com a escolha foi possível promover uma larga discussão e reflexão acerca da temática, discussões transpostas para os textos dos alunos em diversos gêneros, a saber: ilustração, fábula, poesia, memória, artigo de opinião e artigo científico. O projeto consiste em três etapas: formação continuada, produção de textos em sala e submissão ao concurso de produção textual. Houve uma grande aceitação das escolas envolvidas, pois momentos de formação que trabalham com a prática de sala de aula ainda são raros. Neste ano, contamos com a participação de 10 (dez) instituições de ensino e recebemos 450 inscrições para o concurso de redação. Após 08 (oito) meses de trabalho, distribuídos entre planejamento, encontros, produção textual e avaliação, o projeto foi finalizado no dia 13 de setembro, ocasião em que os textos foram premiados. A Escola precisa fomentar ações com vistas a desenvolver a reflexão acerca de temas significantes para a sociedade, competências para que os alunos desenvolvam a criticidade e escrevam textos coerentes. Ações como as propostas pelo projeto “O Poder da Palavra” corroboram para que de fato a escola seja promotora de reflexões, criticidade e dê voz aos alunos para que possam ter vez.
	
	\vspace{\onelineskip}
	
	\noindent
	\textbf{Palavras-chave}: O Poder da Palavra. Formação. Produção Textual.
	
\end{document}
