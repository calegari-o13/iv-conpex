\documentclass[article,12pt,onesidea,4paper,english,brazil]{abntex2}

\usepackage{lmodern, indentfirst, color, graphicx, microtype, lipsum}			
\usepackage[T1]{fontenc}		
\usepackage[utf8]{inputenc}		

\setlrmarginsandblock{2cm}{2cm}{*}
\setulmarginsandblock{2cm}{2cm}{*}
\checkandfixthelayout

\setlength{\parindent}{1.3cm}
\setlength{\parskip}{0.2cm}

\SingleSpacing

\begin{document}
	
	\selectlanguage{brazil}
	
	\frenchspacing 
	
	\begin{center}
		\LARGE CONHECENDO O ESPANHOL\footnote{\textbf{Área de Conhecimento CNPq}: Linguística, Letras e Artes. Fonte de financiamento: Depex, IFRO – Campus Cacoal}
		
		\normalsize
		Maria Eduarda Frank Baldin\footnote{Bolsista (modalidade), eduardafrankbaldin@gmail.com, Campus Cacoal} 
		Vinícius de Oliveira Engelhardt\footnote{Bolsista (modalidade),  vinicius.engelhardt@gmail.com,, Campus Cacoal} 
		Shelly Braum\footnote{Coordenadora, shelly.braum@ifro.edu.br, Campus Cacoal} 
	\end{center}
	
	\noindent A demanda de comunicação com diversas nacionalidades faz necessário dominar uma língua estrangeira. O espanhol tem maior proximidade com o Brasil, pois o país situa-se entre vizinhos e parceiros econômicos que tem a língua de Cervantes como oficial. Embora avanços tenham se estabelecido com a lei 11.161/05, que instituiu o ensino do espanhol nas escolas de Ensino Médio brasileiras, o mesmo não ocorreu no Ensino Fundamental, deixando as escolas facultadas a oferecê-lo. Isso gerou um quadro controverso: a maioria dos alunos chega ao Ensino Médio analfabeta no espanhol, fazendo com que a missão de preparar cientificamente e capacitar para a utilização das tecnologias do conhecimento fique comprometida ou anulada. Portanto, com intuito de mudar um quadro adverso é que este projeto nasceu: realizar um curso básico de idioma em Língua Espanhola a alunos do Ensino Fundamental, para que ao chegarem ao Ensino Médio possam realmente refletir, criticar e transformar a sociedade em que vivem. O projeto foi coordenado pela professora de espanhol do IFRO Campus Cacoal, Shelly Braum, e como bolsistas os alunos do 3º ano do Curso Técnico em Agroecologia integrado ao Ensino Médio, Maria Eduarda Frank Baldin e Vinícius de Oliveira Engelhardt. Foi oferecido na Escola Estadual de Ensino Fundamental e Médio Celso Ferreira da Cunha atendendo 20 alunos, de diferentes idades, entre o 6º e 9º ano do ensino fundamental, em ‘encontros’ semanais, uma vez por semana, na referida escola (a qual foi responsável por escolher os alunos na participação do curso), conduzidos pela coordenadora e pelos bolsistas. As aulas foram expositivas e cooperativas, contando com aparelhagem cedida pelas escolas vinculadas para desenvolver atividades comunicativas que envolviam e priorizavam a aquisição da cultura, língua, fala e escrita. Avaliações orais e escritas eram feitas ao final de cada mês. Semanalmente, os alunos participantes se expressam quanto ao andamento do curso, ao desempenho da coordenadora, dos bolsistas em forma de questionários semiabertos. Concluindo, o projeto “alfabetizou” estudantes dos anos finais do Ensino Fundamental, proporcionando que cheguem ao Ensino Médio com vocabulário e conhecimento básicos de aspectos culturais dos países de fala castelhana, oportunizando ao professor de espanhol do Ensino Médio tempo para discussões mais profundas sobre o mundo hispânico, gerando conhecimento aos alunos. Os resultados obtidos foram satisfatórios, pois alunos que nunca possuíram contato com o espanhol se saíram muito bem com a nova língua, mostrando assim a importância do mesmo no ensino e aprendizagem de Língua Estrangeira. Foi perceptível também a capacidade de cada um, o empenho dos alunos para a aprendizagem do espanhol e o valor sócio-educacional no meio em que os alunos vivem.
	
	\vspace{\onelineskip}
	
	\noindent
	\textbf{Palavras-chave}: Língua. Ensino. Espanhol.
	
\end{document}
