\documentclass[article,12pt,onesidea,4paper,english,brazil]{abntex2}

\usepackage{lmodern, indentfirst, color, graphicx, microtype, lipsum}			
\usepackage[T1]{fontenc}		
\usepackage[utf8]{inputenc}		

\setlrmarginsandblock{2cm}{2cm}{*}
\setulmarginsandblock{2cm}{2cm}{*}
\checkandfixthelayout

\setlength{\parindent}{1.3cm}
\setlength{\parskip}{0.2cm}

\SingleSpacing

\begin{document}
	
	\selectlanguage{brazil}
	
	\frenchspacing 
	
	\begin{center}
		\LARGE RECRIANDO OS LUSÍADAS EM HQ\footnote{Trabalho realizado dentro da área da linguística, letras e artes.}
		
		\normalsize
		Elaine Rodrigues Nichio\footnote{Discente de Língua Portuguesa e Literatura Brasileira do Curso Técnico integrado ao Médio, elaine.rodrigues11@hotmail.com, Campus Colorado do Oeste.}
	\end{center}
	
	\noindent 
	Os Lusíadas de Luís Vaz de Camões trata das viagens dos portugueses por “mares nunca dantes navegados”, sendo uma das características da épica a narração de episódios históricos ou lendários de heróis que possuem qualidade superior. Desta forma, justifica-se no presente projeto de ensino, institucionalizado no IFRO campus Colorado do Oeste, a importância de uma abordagem interdisciplinar, que contou com o aporte do professor de Geografia. O resultado final estará adaptado ao contexto dos adolescentes, visto que histórias em quadrinhos atraem o gosto popular dos mesmos. O projeto está em consonância com os PCNs, pois é perceptível que através da literatura, o aluno trabalha sua liberdade e sua criatividade, o lado da cognição, a percepção e outros aspectos que estejam ligados ao crescimento pessoal do aluno (PCNs, 2006).Sendo assim, objetivou-se proporcionar um outro “olhar” para uma obra tão importante entre os clássicos da literatura, fomentando a leitura e criação artística. A sequência didática ficou disposta na seguinte ordem:Leitura da obra; Fichamento dos dados bibliográficos;Contextualização histórica e geográfica;Pesquisa sobre os elementos mitológicos;Discussão sobre a percepção dos alunos em relação à obra;Aula expositiva sobre os elementos essenciais da história em quadrinhos;Elaboração de histórias em quadrinhos, narrando um dos três maiores episódios da obra: Inês de Castro, O velho do Restelo e o Gigante Adamastor;Confecção de um mural na sala de aula;Apresentação dos grupos: Mitologia Grega, Especiarias, Demarcação das Colônias Portuguesas e Conflitos religiosos.Verificou-se um resultado positivo no decorrer das atividades propostas. Vários foram os fatores que proporcionaram o êxito, como: trabalho interdisciplinar, atividade artística e cultural, em duplas e em grupos maiores, além de resultar na interação da literatura associada a elementos históricos, geográficos e mitológicos. Ficou visível que os aspectos mitológicos da obra podem servir de grande estímulo para os alunos se interessarem pela leitura, pois dos 28 discentes da turma mais de 90\% se mostraram entusiasmados em relação ao assunto, o fato foi constato no sorteio dos grupos que abordaram fatores históricos e geográficos, uma vez que a maioria queria pesquisar sobre mitologia grega.Confeccionar o mural em sala também foi satisfatório. Primeiro foi realizado o fundo do painel e ao passo que os HQs ficaram prontos os próprios alunos socializaram seu trabalho no mural.
	
	\vspace{\onelineskip}
	
	\noindent
	\textbf{Palavras-chave}: Os Lusíadas. Contexto. HQ.
	
\end{document}
