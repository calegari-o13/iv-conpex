\documentclass[article,12pt,onesidea,4paper,english,brazil]{abntex2}

\usepackage{lmodern, indentfirst, color, graphicx, microtype, lipsum}			
\usepackage[T1]{fontenc}		
\usepackage[utf8]{inputenc}		

\setlrmarginsandblock{2cm}{2cm}{*}
\setulmarginsandblock{2cm}{2cm}{*}
\checkandfixthelayout

\setlength{\parindent}{1.3cm}
\setlength{\parskip}{0.2cm}

\SingleSpacing

\begin{document}
	
	\selectlanguage{brazil}
	
	\frenchspacing 
	
	\begin{center}
		\LARGE A PRODUÇÃO DE MATERIAL DIDÁTICO MANIPULÁVEL PARA O ENSINO DE BIOLOGIA\footnote{Trabalho realizado dentro da área de Conhecimento: Educação. Com financiamento do IFRO Campus Vilhena.}
		
		\normalsize
		Luana Fernandes de Almeida\footnote{Bolsista (PROASEN), luana\_nandes@hotmail.com, Campus Vilhena.} 
		Tatiana de Abreu Curado Rezende\footnote{Orientadora, tatiana.rezende@ifro.edu.br, CampusVilhena.} 
	\end{center}
	
	\noindent O Ensino de Biologia no Ensino Médio perpassa inúmeros conteúdos extremamente teóricos, distantes da realidade e cotidiano dos adolescentes, tornando a prática docente cansativa e o aprendizado discente mecanizado e desanimado. O ato de ensinar deve ser visto como tarefa gratificante e motivadora, de forma que os alunos também se sintaminstigados a aprender cada vez mais. A utilização de ferramentas lúdicas para estimular o estudo vai além da diversão que produz, alcançando a desmistificação de temas complexos, como por exemplo, Citologia, Anatomia e Fisiologia Humana. Sabendo da inviabilidade de aulas práticas com microscópios de alta resolução, peças e modelos anatômicos nas escolas brasileiras, faz-se necessária a criação de ferramentas pedagógicas concretas para enriquecer o processo de ensino-aprendizagem. Assim, com a utilização de materiais acessíveis e de baixo custo, como isopor, EVA, canos de PVC e massa de modelar e biscuit, alunos do 2º ano dos Cursos Técnicos Integrados ao Ensino Médio do Campus Vilhena produziram maquetes manipuláveis, podendo ser utilizadas tanto por alunos deficientes visuais quanto videntes. Estas maquetes servem como instrumentos de ensino e quando utilizadas pelo professor durante sua prática docente são capazes de facilitar a percepção do conteúdo através dos estímulos visual e tátil. Para o desenvolvimento deste projeto os estudantes foram orientados pela professora de Biologia de forma que cada material produzido fosse capaz de ampliar o aprendizado através da abstração do assunto abordado. Tais recursos foram utilizados durante as aulas de Biologia e de forma unânime os alunos relataram maior facilidade de compreensão e abstração dos conteúdos ensinados, sendo possível a aproximação ao assunto que anteriormente estava apenas descrito de forma teórica no livro didático e explicado pelo professor no quadro-branco.
	
	\vspace{\onelineskip}
	
	\noindent
	\textbf{Palavras-chave}: Material didático. Biologia. Ensino Médio.
	
\end{document}
