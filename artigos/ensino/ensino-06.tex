\documentclass[article,12pt,onesidea,4paper,english,brazil]{abntex2}

\usepackage{lmodern, indentfirst, color, graphicx, microtype, lipsum,textcomp}			
\usepackage[T1]{fontenc}		
\usepackage[utf8]{inputenc}		

\setlrmarginsandblock{2cm}{2cm}{*}
\setulmarginsandblock{2cm}{2cm}{*}
\checkandfixthelayout

\setlength{\parindent}{1.3cm}
\setlength{\parskip}{0.2cm}

\SingleSpacing

\begin{document}
	
	\selectlanguage{brazil}
	
	\frenchspacing 
	
	\begin{center}
		\LARGE APRIMORANDO MÉTODOS COM MATERIAIS SIMPLES: UMA APRENDIZAGEM DE ÓTICA- REFLEXÃO DA LUZ COM ESPELHOS PLANOS
		Área de Conhecimento CNPq/CAPES: Física
		\footnote{Informações sobre o resumo.}
		
		\normalsize
		Geisiele Aparecida Barbosa\footnote{Bolsista (PIBID), geisiele-barbosa@hotmail.com,Porto Velho – Calama.} 
		Renan Santos Pimentel\footnote{Colaborador (a), renan.pimentel.pvh@hotmail.com,Porto Velho – Calama.} 
		Mauro Guilherme Ferreira Bezerra\footnote{Orientador (a), mauro.guilherme@ifro.edu.br,Porto Velho – Calama.} 
		Fabricio Araújo Sousa\footnote{Co-orientador(a), fisico\_araujo@hotmail.com, Porto Velho – Calama.} 
	\end{center}
	
	\noindent 
	O projeto desenvolvido em sala de aula teve como principal objetivo possibilitar aos alunos do ensino médio de uma determinada escola pública de Porto Velho, um estudo simplificado e prático sobre os fenômenos de reflexão da luz e a importância desse aprendizado para a vida. Foi uma atividade que proporcionou aos alunos identificar conceitos básicos da luz utilizando uma metodologia bem atrativa com uso de slides e práticas experimentais trabalhadas em grupos, foi possível apresentar aos alunos de forma prática como ocorre a primeira e a segunda lei da reflexão da luz usando materiais simples. Para que os alunos conseguissem visualizar estas leis foram confeccionados por eles um experimento simples utilizando um pequeno espelho de maquiagem, cartão opaco, folha de papel, transferidor e um laser pequeno com cor de raio de propagação opcional. Fazendo uso de uma sequência didática os alunos conseguiram identificar como ocorre a reflexão da luz e compreender as leis formuladas para este fenômeno ocorrer. Ao continuar com as atividades, foi utilizado um exemplo que os alunos tinham em mãos o espelho plano do primeiro experimento, a abordagem do assunto ocorre de uma forma simples fazendo apenas duas perguntas: O que veem no espelho? Como surge a imagem no espelho? Para responder as pergunta de uma forma bem clara foi utilizado alguns slides, mostrando de uma forma bem visual que quando os raios de luz que parte do objeto e são incididos no espelho ocorre um cruzamento no prolongamento desses raios no espelho formando assim uma imagem virtual. Para que os alunos conseguissem entender de uma forma pratica, foram confeccionados três experimentos utilizando espelhos planos. Tais experiências desenvolvidas durante o projeto levaram grande conhecimento aos alunos, de acordo com o levantamento dos questionários aplicados em cada atividade. Contudo, aprenderam de forma significativa, como a luz o fenômeno mais importante da ótica se comporta em determinados casos e meios. Participaram deste projeto alunos do 2° ano do ensino médio, onde a maioria acreditava que a luz era um fator relevante a ser estudado, reconstituindo assim o conhecimento a partir das atividades desenvolvidas.
	
	\vspace{\onelineskip}
	
	\noindent
	\textbf{Palavras-chave}: Aula Prática. Ensino de Física. Reflexão da Luz. 
	
\end{document}
