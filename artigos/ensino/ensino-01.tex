\documentclass[article,12pt,onesidea,4paper,english,brazil]{abntex2}

\usepackage{lmodern, indentfirst, nomencl, color, graphicx, microtype, lipsum}			
\usepackage[T1]{fontenc}		
\usepackage[utf8]{inputenc}		

\setlrmarginsandblock{2cm}{2cm}{*}
\setulmarginsandblock{2cm}{2cm}{*}
\checkandfixthelayout

\setlength{\parindent}{1.3cm}
\setlength{\parskip}{0.2cm}

\SingleSpacing

\begin{document}
	
	\selectlanguage{brazil}
	
	\frenchspacing 
	
	\begin{center}
		\LARGE ANÁLISE QUANTITATIVA E QUALITATIVA DOS RESIDUOS RESIDENCIAIS E COMERCIAS RECICLADOS NA CIDADE DE JI-PARANÁ –RO\footnote{Projeto de ensino desenvolvido com o 4ºAnoB do curso técnico em florestas, no IFRO campus Ji-Paraná na disciplina de matemática no ano de 2016.}
		
		\normalsize
		Aline Soares Dormiro Claudino\footnote{PIBIC, aline.sclaudino@gmail.coml, Campus Ji-Paraná.} 
		Maria Clara Da Costa Fernandes\footnote{mariaclarafernandes598@gmail.com,Campus Ji-Paraná.}
		Érica Patrícia Navarro\footnote{erica.navarro@ifro.edu.br , Campus Ji-Paraná.}
		Andreza Mendonça\footnote{andreza.mendonca@gmail.com. Campus Ji-Paraná.}
		Alice SperandioPorto\footnote{alice.porto@ifro.edu.br, Campus Ji-Paraná.}
	\end{center}
	
	\noindent No intuito de avaliar a quantidade e qualidade dos resíduos domésticos e comerciais reciclados pelos catadores da Cooperativa de Catadores de Materiais Recicláveis de Ji-Paraná (Coocamarji), foi realizado o levantamento dos materiais reciclados na cidade e seu destino final. Para o levantamento de dados foi realizado uma entrevista com um dos catadores e com o presidente da Cooperativa, procurando identificar a quantidade de resíduos que são reciclados e o destino final destes produtos. Assim foi feito levantamento referente aos resíduos gerados na cidade de Ji-Paraná-RO que são encaminhados para o aterro.A cooperativa Coocamarji localiza-se no Km 11 Linha 11 em Ji-Paraná - RO, foi criada em 2010, com o objetivo de buscar um melhor valor nos materias recicláveis, hoje a cooperativa é constituída por 30 catadores. Os resíduos recicláveis são acumulados em fardos e vendidos para empresas situadas no estado de São Paulo e em Rondônia, sendo para a cidade de Vilhena feita a venda do alumínio e do cobre e estes dão o destino final do material exportando-o para a China. Os benefícios vindos da reciclagem são diversos, não apenas relacionado ao meio ambiente, mas também aos trabalhadores, sejam estes diretos ou indiretos. Todos os dias são coletados aproximadamente 100000 toneladas de lixo. Em média o Brasil recicla 3\% dos resíduos e Ji-Paraná recicla 1\%, assim para atingir a média nacional é preciso que a quantidade de resíduos reciclados passe de 1500 kg/dia para 3500 Kg/dia. Em Ji-Paraná o processo de reciclagem é precário, com falta de estrutura os catadores passam o dia no sol e os poucos que trabalham na sombra passam o dia pisoteando restos de lixo e os sumos que estes expelem quando são prensados. Projetos de conscientização são feitos pela cooperativa para mobilizar a população a respeito da importância da separação do lixo doméstico, porém os resultados dessa sensibilização ainda não são sentidos pelos catadores. Essa mudança exige um esforço pessoal de cada cidadão, sendo primordial para que em um futuro próximo, não só a qualidade de trabalho dos catadores melhore, mas também a qualidade ambiental da cidade de Ji-Paraná.
	
	\vspace{\onelineskip}
	
	\noindent
	\textbf{Palavras-chave}: Resíduos. Recicláveis. Sustentável.
	
\end{document}
