\documentclass[article,12pt,onesidea,4paper,english,brazil]{abntex2}

\usepackage{lmodern, indentfirst, color, graphicx, microtype, lipsum}			
\usepackage[T1]{fontenc}		
\usepackage[utf8]{inputenc}		

\setlrmarginsandblock{2cm}{2cm}{*}
\setulmarginsandblock{2cm}{2cm}{*}
\checkandfixthelayout

\setlength{\parindent}{1.3cm}
\setlength{\parskip}{0.2cm}

\SingleSpacing

\begin{document}
	
	\selectlanguage{brazil}
	
	\frenchspacing 
	
	\begin{center}
		\LARGE RELATO DE EXPERIÊNCIA DA IMPLANTAÇÃO DO MODELO ENSINO BASEADO EM PROBLEMAS OU PROJETOS\footnote{Trabalho realizado dentro da área de Conhecimento CNPq/CAPES: Ensino-Aprendizagem.}
		
		\normalsize
		Geovane C. S. Pagani\footnote{Discente, geovane\_pagani@hotmail.com, CampusIFRO Porto Velho Calama.} 
		Shalom M. da Silva\footnote{Discente, shalombio@hotmail.com, Campus IFRO Porto Velho Calama.} 
		Aline A. de Souza\footnote{Discente, aline.alkimin@hotmail.com, CampusIFRO Porto Velho Calama.} 
		Delmo L. de Araújo\footnote{Discente, delmolima@gmail.com, CampusIFRO Porto Velho Calama.} 
		Júlio de S. Marques\footnote{Discente, julio.m@hotmail.com.br, CampusIFRO Porto Velho Calama.}
	\end{center}
	
	\noindent Um dos grandes desafios da educação brasileira hoje é saber quais métodos de ensino que podem ser utilizados para uma aprendizagem significativa e capacitar profissionais mais qualificados para a resolução de problemas reais. Neste sentido, este estudo visa descrever as experiências vivenciadas no curso de Pós-Graduação Lato Sensu em Gestão Ambiental, Instituto Federal de Educação, Ciência e Tecnologia de Rondônia – IFRO. Esse curso, recentemente, adotou uma metodologia diferenciada denominada de problem-basedlearning e project-basedlearning (aprendizagem baseada em problemas ou projetos). Para tanto, os problemas fictícios ou reais da comunidade são o ponto de partida para o desenvolvimento da prática docente e do aprendizado do estudante. Acredita-se que a experiência tem mostrado a viabilidade desse método, a começar pela disposição da mobília na sala de aula, que são dispostas por mesas redondas facilitando a convivência e o trabalho colaborativo. Nesse modelo de ensino, os alunos são divididos em pequenos grupos separados por afinidades ou área de atuação observando a composição diversificada de profissionais. O time EcoVita, por exemplo, é composto por dois gestores ambientais, uma bióloga, um engenheiro florestal e um psicólogo. Esses grupos funcionam como pequenas empresas empreendedoras que busca resolver problemas presentes na nossa comunidade, aplicando o conhecimento de cada disciplina aos projetos. Esse método já foi aplicado em três módulos, a saber: fundamentos de gestão ambiental, normas e técnicas para elaboração de trabalho cientificam e fundamentos de direito e legislação socioambiental. Todos os módulos foram desenvolvidos com base nessa nova abordagem de ensino. Defende-se, sob a perspectiva do estudante, que esse modelo de aprendizado é importante para tornar os futuros especialistas em gestão ambiental, mais críticos, proativos e preparados para as soluções de problemas, sobretudo criar alternativas para o desenvolvimento sustentável da região, de modo a fomentar a economia, gerar emprego e preservar os recursos ambientais.
	
	\vspace{\onelineskip}
	
	\noindent
	\textbf{Palavras-chave}: Pós-graduação. Gestão Ambiental. Ensino e Aprendizado.
	
\end{document}
