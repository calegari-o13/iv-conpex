\documentclass[article,12pt,onesidea,4paper,english,brazil]{abntex2}

\usepackage{lmodern, indentfirst, nomencl, color, graphicx, microtype, lipsum, textcomp}			
\usepackage[T1]{fontenc}		
\usepackage[utf8]{inputenc}		

\setlrmarginsandblock{2cm}{2cm}{*}
\setulmarginsandblock{2cm}{2cm}{*}
\checkandfixthelayout

\setlength{\parindent}{1.3cm}
\setlength{\parskip}{0.2cm}

\SingleSpacing

\begin{document}
	
	\selectlanguage{brazil}
	
	\frenchspacing 
	
	\begin{center}
		\LARGE ANALISE SOCIOAMBIENTAL DA\\PRAÇA DOS MIGRANTES, JI-PARANÁ/RO
		
		\normalsize
	Dayane Rodrigues Raasch1, Elizeu Ribeiro da Silva Neto \footnote{Discente do curso técnico em florestas – IFRO. e-mail: dayane.raasch@gmail.com} 
	Elizeu Ribeiro da Silva Neto\footnote{Discente do curso técnico em florestas- IFRO. e-mail: elizuerib@gmail.com} 
	
	\end{center}
	
	% resumo em português
	\begin{resumoumacoluna}
		O presente trabalho apresenta o estudo geográfico da paisagem urbana na área da Praça dos Migrantes na cidade de Ji-Paraná/RO. Como resultado deste trabalho, verificou-se que a população que frequenta a praça dos migrantes desejaria algumas mudanças no ambiente, sendo minoria moradora do bairro. A paisagem da praça é composta por elementos naturais e artificiais, como brinquedos que são feitos a partir do reaproveitamento de pneus, um ato de educação ambiental, assim verificou-se pontos positivos e negativos na utilização dos pneus e também com a pouca quantidade de arvores presentes no ambiente exposto. A praça assim como todo espaço público, ao mesmo tempo em que propicia a integração das pessoas, com lazer, atividade física entre outras atividades, pode afastar as pessoas desse espaço por causa de algumas deficiências bem notáveis noespaço.
	
		
		\vspace{\onelineskip}
		
		\noindent
		\textbf{Palavras-chave}: Pneus. Espaço Público. Lazer.
	\end{resumoumacoluna}
	
	\section*{Introdução}
	
	O uso das praças, nas últimas décadas tem mudado significativamente em todo o Brasil. O rápido crescimento das cidades brasileiras a partir da segunda metade do século XX foi fundado em uma concepção de desenvolvimento urbano que, em geral, desconsiderou as condições ambientais preexistentes.
	
	Na cidade de Ji-Paraná, a Praça dos Migrantes é o objeto de estudo da presente pesquisa, onde a classificação conceitual paisagem, foi utilizada como uma leitura geográfica na perspectiva do espaço público. Sendo assim, analisando seus pontos positivos e negativos, como por exemplo a utilização de pneus para fazer brinquedos para a praça, se o espaço é adequado para atividades recreativas e outros fins.
	
	Este trabalho justifica-se pelo reduzo interesse de pesquisas acerca dessa praça e, principalmente, de espaços públicos de fato como palco e cenário da ação, mobilização e articulação por parte da população.
	
	
	\section*{Material e Método}
	
	Segundo IBGE (apud MACARO, 2013, p. 20-21), o município de Ji-Paraná está localizado, na região Norte do Brasil, na porção centro-leste do estado de Rondônia. Com coordenadas geográficas entre as latitudes 8°22’00”S e 11°11’00”S e longitudes 61°30’00”W e 62°22’00”W, com área correspondente de 6.922 km2. “O perímetro urbano do município de Ji-Paraná localiza-se na porção sul, e é cortada pela principal rodovia do Estado, a BR-364”. (MACARO, 2013,p.20).
	
	A área de estudo foi na Praça dos Migrantes (Prefeitura de Ji-Paraná/RO, 2015) soba coordenada geográfica latitude 10°52’34,8”S e longitude 61°57’43,8”W, com altitude 163 metros, figura 2.
	
	A metodologia para atingir os objetivos propostos da referida pesquisa tem como método de abordagem um estudo indutivo e quantitativo, por se tratar de um estudo que é o caso do espaço da Praça dos Migrantes.
	
	E na metodologia usada neste trabalho, os alunos, observaram com outro olhar a paisagem que eles já conheciam, para fazer a leitura crítica e desenvolver a pesquisa.
	
	O início do trabalho realizado com uma visita a campo, área de estudo, Praça dos Migrantes com objetivo de fazer o reconhecimento do local e registrar os elementos da paisagem. Para fazer o registro da paisagem da Praça foi usada uma geotecnologia, uma câmera fotográfica digital da marca Sony e modelo Cyber-shot DSC-H7 8.1.
	
	Outro recurso utilizado foi um questionário, com objetivo de formular hipóteses reais sobre a população que faz uso do espaço público da Praça dos Migrantes em Ji-Parana/RO.
	
	Foram feitas duas visitas há praça. A primeira para fotografar os elementos que fazem parte da paisagem da praça. E a segunda para aplicar o questionário, fazer entrevista com a população, sendo um total de 40 pessoas que estavam presentes na Praça.
	Na primeira visita à Praça, foi possível perceber e identificar elementos naturais e artificiais. Nos elementos naturais a existência de árvores de pequeno, médio e grande porte.
	
	Na paisagem da praça, no entorno, é cercada de residências, possui um ginásio de esportes, uma área de ginástica, um playground com brinquedos confeccionados por pneus e um centro cultural. Todas as sextas-feiras é realizado na praça, no período noturno, uma feira de produtos artesanais e alimentícios. Esse diferencial motivou a construção da presente pesquisa, além de sua representatividade para a população da cidade onde está inserida.
	
	\section*{Resultados e Discussão}
	
	Dos entrevistados trinta e sete pessoas eram moradores da cidade de Ji-Paraná e destas oito pessoas moram no Bairro Jardim dos Migrantes.
	
	Quando perguntados sobre a importância do da praça como um espaço público, trinta e seis pessoas acharam importante e apenas quatro pessoas não consideram a praça um espaço público importante.
	
	Sobre os imigrantes que fazem uso da praça, quinze entrevistados vieram após o ano de 2000. Doze entrevistados vieram entre a década de 90 até o ano de 2000. Oito entrevistados vieram entra as décadas de 70 e 90. E dois entrevistados vieram entra as décadas 50 e 70. Três entrevistados vieram entre as décadas de 30 e 50.
	
	No tema uso da praça, dezessete pessoas responderam que frequentam a praça para socializar com amigos. Dezesseis pessoas responderam que usam a praça para atividade esportiva. Outras quatro pessoas declararam trabalhar na praça e outras três pessoas afirmaram usar para fins recreativos.
	
	Quando perguntados sobre os elementos presentes na praça, dezessete pessoas responderam que utilizam os bancos presentes na praça. Dez pessoas utilizam os brinquedos. Cinco pessoas declararam utilizar as arvores. E  oito pessoas declararam utilizar os aparelhos de exercícios físicos.Dos entrevistados nove pessoas consideram a presença e a quantidade de árvores existente na praça importante e satisfatório. Trinta pessoas consideram as árvores importantes, mas acham que poderia ter mais arvores e plantas na praça, não houve nenhuma pessoa que discordou desse item.
	
	Trinta e nove entrevistados concordaram que as árvores existentes na praça propiciam algum benefício para a população humana e para as aves que habitam a praça. Apenas uma pessoa declarou que as árvores não trazem benefício.No tema satisfação sobre o estado de conservação da praça, onze entrevistados disseram estarem satisfeitos. Quinze respondeu que poderia ter
	melhorias na praça. E catorze pessoas declararam não estarem satisfeitos.
	
	\section*{Conclusões}
	
	A conclusão final deste trabalho aponta pontos positivos e negativos. A praça assim como todo espaço público, ao mesmo tempo em que propicia a integração das pessoas, com lazer, atividade física entre outras, pode afastar as pessoas desse espaço. Informações adquiridas com os relatos dos entrevistados confirmaram que pessoas fazem uso da praça para consumir substancias ilícitas. A estrutura da praça já sofreu depredação e fato como a presença de lixo na grama, deixa explicita que mesmo tendo lugar específico para acondicionar resíduos sólidos, parte da população não tem preocupação em preservar e conservar a limpeza da praça.
	
	\section*{Instituição de Fomento}
	
	Instituto Federal de Ciência e Tecnologia- IFRO / Campus Ji-Paraná, forneceu matérias para a pesquisa.
	
	\sloppy
	
	\section*{Referências}
	
	\noindent LIMA, A. G. A bacia hidrográfica como recorte de estudos em geografia humana, GEOGRAFIA – Universidade Estadual de Londrina, Departamento de Geociências - v. 14,  n. 2, jul./dez.2005.
	
	
	\noindent MACARO, L. A. M. Geotecnologias aplicadas à caracterização da qualidade ambiental da bacia hidrográfica do igarapé pintado. 2013. 37 f. Monografia – Departamento de Engenharia Ambiental, Universidade Federal de Rondônia, Campus de Ji-Paraná - RO.
	
	
	\noindent MEDINA, N. M; SANTOS, E. C. Educação Ambiental: Uma metodologia participativa de formação. Petrópolis, Vozes, 2008.
	
	
	\noindent Normas e padrões para trabalhos acadêmicos e científicos da unoeste. Universidade do Oeste Paulista – UNOESTE, Presidente Prudente, São Paulo. Disponível em: < https://unoeste.br/site/biblioteca/documentos/Manual-Normalizacao.pdf> Acesso em: 03 de agosto de 2015.
	
	
	\noindent SCHIER, R. A. Trajetórias do conceito de paisagem na Geografia. R.RA’E GA, Curitiba: UFPR,	nº.	7,	p.	79-85,	2003.	Disponível	em:
	<http://ojs.c3sl.ufpr.br/ojs2/index.php/raega/article/viewFile/3353/2689> Acesso em: 10 ago. 2015.
	
	
\end{document}
